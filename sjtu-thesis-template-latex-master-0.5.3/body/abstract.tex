%%==================================================
%% abstract.tex for SJTU Master Thesis
%% based on CASthesis
%% modified by wei.jianwen@gmail.com
%% version: 0.3a
%% Encoding: UTF-8
%% last update: Dec 5th, 2010
%%==================================================

\begin{abstract}
近些年来,越来越多的混合现实设备被研制出来,不论是手持式还是佩戴式都层出不穷,而对于其交互手段、交互界面与应用场景均尚在探索阶段。
其中混合现实设备上的交互界面仍以二维界面为主,虽然简洁明晰但却与用户所处的三维空间不相符合;
此外,对于此类设备的交互手段研究相当丰富,一部分以肢体、声音及触觉等人体感知的方法进行交互,另一部分设计出增强媒介如带有图案(pattern)的笔等来进行交互,
方式不同而关键在于是否适合所设计的应用环境及用户体验是否良好;
多数增强与混合现实应用都具有导入模型的功能,导入模型虽然方便却不具有个性化,
针对用户在增强与混合现实的应用场景下进行模型创建的研究也很多,如何避免用户自身的建模能力不足又同时赋予用户足够的自由度则是主要的难点。
本文围绕这些问题,主要针对(1)混合现实眼镜的制作;(2)三维交互界面;(3)以手势为主的交互手段;以及(4)应用性广泛的徒手三维建模场景进行设计与评估。

本文完成的主要工作成果有:

(1)首先制作一个视频透视的混合现实眼镜,为之后的研究工作做铺垫。

(2)在自制的混合现实眼镜下设计了调色盘菜单,并针对不同的思维模式设计了三种布局,分别是手掌召唤式菜单、目标跟踪式菜单及屏幕固定式菜单。

(3)针对虚拟物体的基本操控设计了以手势为主配合头部动作的交互方式,并为了用户更好的体验增加了头部约束、骨骼小球和顶置提示板作为辅助功能帮助用户交互。

(4)针对增强与混合现实应用中常用的模型使用,设计其独特的建模方式,利用该自制设备释放双手的特性,并参考双手交互的三项原则,且针对前人工作中存在的不足提出了改进方法,包括对用户自身建模能力的优化、支持无缝衔接建模时不同状态的切换,和提高用户创建定制模型的自由度,设计并实现了徒手三维建模的应用场景。

每一部分的工作都通过详细的用户实验进行评估,分别验证三维界面、三维操控与徒手三维建模的可行性与易用性。
从用户的反馈与记录的数据中可以初步得出结论,本工作设计的调色盘菜单与基于手势的交互可以应用在眼镜类的混合现实设备上,并且徒手三维建模的应用场景完成度与可用性都较好。

  \keywords{
  %\large 
  \songti\zihao{-4} 混合现实眼镜,手势交互,三维建模,三维菜单}
\end{abstract}

\begin{englishabstract}

In recent years, more and more Mixed Reality(MR) devices were developed, no matter hand-held devices or wearable devices.
While when it comes to the interactive styles, user interfaces and applications of these devices, the researches are still ongoing.
The user interfaces of MR devices are still 2D for most. Although it's clear and simple, it doesn't match the 3D space where users are;
In addition, there are plenty of researches on interaction styles of these devices. Some of them interact through body, audio, tactile sensation and other human sense, while others are based on Augmented Reality(AR) agent like pen with pattern.
The methods are all different, and what matters is whether it's suitable to application and the user experience.
Apart from that, many AR and MR application has the function of importing models.
It's convenient to import models, but it's not personalize.
There are many researches on creating models in AR and MR application either.
The main difficulties are how to avoid the shortness of people's modeling ability and how to maximize people's freedom to create.
This work focused on these problem:
(1)the fabrication of a MR glasses,
(2)the 3D user interface,
(3)the user interaction style based on gestures,
and (4)3D modeling with bare hands, which is widely used in applications.
Evaluation are presented after design.

The followings are main work in this paper:

(1)First, a video see-through MR glasses is made by ourselves, which is used for the rest of our work.

(2)Palette-like 3D menus and three kinds of placement are designed and implemented based on MR glasses to meet different ways of thinking, such as palm-based menus, object-tracked menus and screen-fixed menus.

(3)3D basic manipulation method mainly based on gestures are designed and implemented, and head motion is supported for assistance.
In addition, to improve the user experience, different assistance methods including head constraints, skeleton balls and board are added to support user's manipulation.

(4)At last, a new method of creating 3D models, which is familiar in all kinds of AR and MR applications is presented.
We created our own principles for bimanual interaction on 3D modelling based on the hands-free devices and former principles
and the 3D modeling application with bare hands.
We pinpoint the shortage of previous work and improve people's modelling result even they are bad model creators by a seamless modelling environment among different status switch, and enough freedom for people to create their own models. 

After all, we evaluate the feasibility and usability of every part of the work by experiments, which includes user interface, manipulation and 3D modelling.
We could safely draw the conclusion on user's feedback and data collected in experiments that the palette-like menus and interactive styles for manipulation are suitable under MR glasses, and the application of 3D modelling by two bare hands has got nice feedback as well.

  \englishkeywords{\large Mixed Reality glasses, gesture interaction, 3D modeling, 3D menus}
\end{englishabstract}
