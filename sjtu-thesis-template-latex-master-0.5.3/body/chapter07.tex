%%==================================================
%% chapter03.tex for SJTU Master Thesis
%% Encoding: UTF-8
%%==================================================

\chapter{结论与未来展望}
\label{chap:conclusion}

\section{主要工作与创新点}
%本章节将主要总结完成的所有工作以及在系统设计与实现中蕴含的创新点。
\subsection{主要工作}
本文所设计系统的主要贡献分为三个部分,首先是对于混合现实眼镜下三维菜单的设计、接着是三维操控方法的设计,最后是徒手三维建模方法的设计。
其中包括对于混合现实眼镜ARGlasses的制作、不同设备间的标定、调色盘型菜单的设计与实现、单手和双手三维操控方法的设计与实现、结合用户反馈的辅助设计、双手三维建模的交互理念、自由虚拟网格平面的设计以及双手职责分离的概念实现。
\begin{enumerate}
\item 混合现实眼镜的制作\hfill\\
本文所设计的系统建立在该眼镜的制作上,故而先设计并制作硬件,其配置从早期的笨重到最新版本的轻巧。
最开始有USB相机、Leap Motion和Oculus VR组成,为了固定USB相机设计了专有的固定夹板,这也是质量偏重的源头之一。
而后由于视野过于狭窄增加了广角镜头,将USB相机拆解至电路板层级成功安装上了广角镜头,角度也增加一倍不止。
最后由最新的Leap Motion SDK版本提供了图像接口,直接由其得到图像而舍弃原有的USB相机,完成设备的改进。

由于硬件设备的不断更新,设备之间的标定方法也一直在变动。
最初是双目相机的标定,利用棋盘格进行各自的内参和互相间的外参标定;然后是和Leap Motion之间的粗略标定,
而这个粗略标定也给后期的一些实现增加了复杂度。受限于Leap Motion只能检测手部骨骼的特质而在相同的位置安置了棋盘格和手指,对点对进行了计算得到了变换矩阵。
再之后由于视野太小而新增了广角镜头,于是转为广角镜头进行广角标定,反扭曲广角镜头拍摄到的图像,再进一步进行和Leap Motion的标定。
直到最新版本Leap Motion开放图像接口,于是输入设备减少为一个,无需与其他设备进行标定。

\item 三维操控及辅助方法的设计\hfill\\
同样,本文所设计的系统针对单手和双手分别设计了不同的交互手段,配合三维应用空间设计三维交互手段。单手交互更接近于触屏操控,屏幕为空中的任意平面;
双手交互无需选择命令,因而可以同时进行多项操控。
之后的实验环节除了对交互手段的可行性分析外同样比较了单手交互和双手交互的利弊。

在完成了混合现实眼镜下的的可行性分析后,增加了顶置提示板、骨骼小球和头部约束,来针对实验者的表现情况和反馈。
每一项新增的设计都进行了相同任务的评测,在客观的时间记录和主观的用户体验下一一分析。

\item 双手交互设计原则的更新 \hfill\\
根据\ref{sec:related-shou}节讨论过的三项双手交互重要原则结合现在的应用环境重新设计了本文系统的双手交互设计原则。
分别是由右及左,对称的工作规模和右手先行。
之后对于用户进行空间设计的具体操作均考虑了这三个基本原则,并对于这三条原则在建模中的实用性进行了评估。

\item 可行性与易用性评估 \hfill\\
本文首先评估了所设计的三维菜单与布局在混合现实眼镜下的可行性,
然后评估了结合头部约束、顶置提示板、骨骼小球的单手及双手操控交互的易用性,
最后评估混合现实眼镜下的徒手建模应用的可行性。

\end{enumerate}

\subsection{创新点}
本次工作的创新点主要为
\begin{enumerate}
\item 三维菜单的设计\hfill\\
本文所设计的系统设计了三维菜单,针对增强现实所处的三维操作空间设计独有的三维交互界面,同时借鉴了画家画画的寓意将一款菜单设计在手掌之中,而形状和人手本身,调色盘接近,视为圆形。同时考虑到不同的需求设计了三种布局,分别是手掌召唤式菜单、目标跟踪式菜单和屏幕固定式菜单。并对三种布局的偏好通过实验进行了详细的评估。
\item 自由虚拟网格平面 \hfill\\
本文所设计的系统基于H{\"u}rst等人的虚拟网格平面想法作了进一步深化\upcite{TrackingBased},
帮助用户既能创建具有自己特色的独一无二的模型,又能尽量消除普通人并不擅长画画的缺点,
加强了其的可操作度,针对朝向和大小进行了特别的操控设计。
同时由于网格交点的限制,用户只能在交点上建模,保证了建模质量,而在实战实验中也可以看到实验者们创作的模型,验证了其对用户自创模型的自由度和基本的质量保障。

\item 职责分离的双手\hfill\\
本文所设计的系统将左右手的职责分开,分别为绘制之手和操控之手,而在建模途中两只手可以同时操作,达到在工具切换过程中无需中断建模过程的效果,实现无缝状态切换与无间断建模的目的。
虽然刚开始用户略显不适应,但几次训练之后便可以熟练操控一边建模一边调整工具修改网格属性的特性。其特色也在实验部分被充分验证。

\end{enumerate}

\section{后续研究工作}
后续研究工作分为两个部分:
%\begin{enumerate}
%\item[其一在于扩大模型创建的功能]\hfill \\
其一在于扩大模型创建的功能。
目前的建模都是做加法,不断地增加新的内容,而不支持诸如洞之类的形态,因而对于挖洞功能的支持将极大地丰富模型的多样性。

%\item[其二在于修改已经建好的模型]\hfill \\
其二在于修改已经建好的模型。
如果用户对于已经创建好的模型想要进一步细化,如调整旋转角度,平滑边线等,本文所设计的系统现下是不支持的,因而在后续工作中继续研究如何修改模型这一功能将使用户可以更好地创建自己的模型。
%\end{description}